\documentclass[oneside,final,14pt]{extreport}
\usepackage[utf8]{inputenc}
\usepackage[russianb]{babel}
\usepackage{vmargin}
\usepackage{amsmath}
\usepackage{amsfonts}
\usepackage{amssymb}
\usepackage{graphicx}
\usepackage[usenames]{color}
\setpapersize{A4}
\setmarginsrb{2cm}{1.5cm}{1cm}{1.5cm}{0pt}{0mm}{0pt}{13mm}
\usepackage{indentfirst}
\sloppy



\begin{document}

\section* {
\begin{centering}
	Вычислительные Методы\\
	Отчёт по лабораторной работе №5\\
\end{centering}
}


\begin{center}
	Выполнил студент группы КЭ-201 Гордеев Александр\\
	Вариант № 8\\
\end{center}


\begin{center}
	\begin{tabular}{|l|l|l|l|l|l|l|l|}
		\hline
		X & 2,0 & 2,3 & 2,5 & 3,0 & 3,5 & 3,8 & 4,0 \\
		\hline
		f(x) & 5,848 & 6,127 & 6,300 & 6,694 & 7,047 & 7,243 & 7,368\\
		\hline
	\end{tabular}
\end{center}

$x = 3.75$

$\omega_{n+1}(x) = 0.00743408$

$Ln(x) = 7.21109152$

$r_n(x) = 0.00001087$

\end{document}
